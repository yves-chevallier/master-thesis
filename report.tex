%% 
%% Optical Tweezer in a Volume Holographic Application
%% @author Y.Chevallier <me@yves-chevallier.com>
%% @language LaTeX2e
%% @date Jan 2012
%% --------------------------------------------------------------------------
\documentclass{kepfl}
\usepackage{glossaries}
\makeglossaries

\lstset{%
language=C,
breaklines=true, 
breakatwhitespace=false, 
basicstyle=\footnotesize}

\begin{document}

%% Define title page
%% --------------------------------------------------------------------------
\pagenumbering{empty}
\date     {\today}
\author   {Yves Chevallier}
\title    {Holographic Optical Tweezers for a \\Volume Hologram Imaging application}
\subtitle {Masters thesis 2011-2012}
\teacher  {Prof. C. Moser (EPFL)\\Prof. G. Barbastathis (MIT)\\Prof. Yuan Yao (National Taiwan University)}
\maketitle
\newpage

\frontmatter 

%% Acknowledgements
%% --------------------------------------------------------------------------
\begin{acknowledgements}
	\input{0.acknowledgements}
\end{acknowledgements}
\clearemptydoublepage

%% Abstract page
%% --------------------------------------------------------------------------
\begin{abstract}
	% English
	\lipsum[1]

	\vspace{1em}
	\asterism
	\vspace{1em}

	% French
	\lipsum[2]
\end{abstract}

%% Table of contents, list of figures and tables and glossary
%% --------------------------------------------------------------------------
\clearemptydoublepage
\tableofcontents
\clearemptydoublepage
\pagenumbering{arabic}

%% Content
%% --------------------------------------------------------------------------
\mainmatter
\chapter{Introduction}
SMART developped VHIS 
Wants to do Tomography of cells, they needs to rotate and move the sample with a good accuracy. 
Focused laser beam can trap small particles. This technique called Optical Tweezers and pionered by ashkins in 1970 is today well known. 
If few industrials systems exists, this area is still active in research. 
Censam never used optical tweezers and want to assess the feasability of using optical tweezer to rotate cells for doing VH TIE Tomography

This masters project was conducted at S.M.A.R.T (Singapore MIT Alliance for Research and Technology) and MIT Boston. 
In collaboration with Yuan Luo from University \ldots and supervised by Prof. George Barbastathis.

\subsection{Context}
Tomography is a well known imaging technique that allows three-dimentional reconstruction of object by sectionning the object. 

Projected tomography such as Computed tomography (CT) uses for instance X-ray to create projection.
The sample is need to be rotated along a fixed axis. 

Electron Tomography is almost the only method available to image small structures such as bacterias or biological cells. 
However projected images are created by the absorption of electrons in the material. Cells are mainly transparent and it is then difficult to get a contrasted tomogram
SMART wants to leverage its volume holography system with the transport intensity equations to get three dimensional images from biological phase objects. 

Within the need of rotating the sample along a fixed axis, very precise rotation stage is needed and high magnification reduce the working distance. 

Some commercial systems require to place the sample in a small glass tube which can be an issue for certain type of samples. 

The main idea is to use the base of a regular brightfield microscope and use optical tweezers to hold and rotate the sample.

\subsection{Project goal}
- optical tweezers system
- build an optical tweezers setup
- allow multitrapping
- allow rotation of the sample
- merge the optical tweezers setup with the VHIS
- get images and restults
- propose futur actions

\chapter{Optical Trapping}
\chapter{VHIS}
\chapter{Holographic Optical Tweezers}
\chapter{Hologram Generation}
\chapter{Optical Setup}

\chapter{Microstructures}


\section{Thin Lens}

\begin{equation}
\Delta \varphi (x,y) = \Delta {\varphi _0} - {R_1}\left( {1 - \sqrt {1 - \frac{{{x^2} + {y^2}}}{{{R_1}^2}}} } \right) + {R_2}\left( {1 - \sqrt {1 - \frac{{{x^2} + {y^2}}}{{{R_2}^2}}} } \right)
\end{equation}

\begin{equation}
\Delta \varphi (x,y) \approx \Delta {\varphi _0} - \frac{{{x^2} + {y^2}}}{2}\left( {\frac{1}{{{R_1}}} - \frac{1}{{{R_2}}}} \right)
\end{equation}


\begin{equation}
	\begin{array}{c}
L(x,y) = {e^{ - i\frac{{2\pi }}{\lambda }\left( {n\Delta \varphi (x,y) + \Delta {\varphi _0} - \Delta \varphi (x,y)} \right)}}\\
 = {e^{i\frac{{2\pi }}{\lambda }\Delta {\varphi _0}}}{e^{ - i\frac{{2\pi }}{\lambda }(n - 1)\Delta \varphi (x,y)}}\\
 = {e^{i\frac{{2\pi }}{\lambda }\Delta {\varphi _0}}}{e^{ - i\frac{{2\pi }}{{\lambda f}}({x^2} + {y^2})}}
\end{array}
\end{equation}


\begin{equation}
\frac{1}{f} = (n - 1)\left( {\frac{1}{{{R_1}}} - \frac{1}{{{R_2}}}} \right)	
\end{equation}

\begin{equation}
L(x,y) = {e^{ - i\frac{{2\pi }}{{\lambda f}}({x^2} + {y^2})}}
\end{equation}

\begin{equation}	
% MathType!MTEF!2!1!+-
% feaagCart1ev2aaatCvAUfeBSjuyZL2yd9gzLbvyNv2CaerbuLwBLn
% hiov2DGi1BTfMBaeXatLxBI9gBaerbd9wDYLwzYbItLDharqqtubsr
% 4rNCHbGeaGqiVu0Je9sqqrpepC0xbbL8F4rqqrFfpeea0xe9Lq-Jc9
% vqaqpepm0xbba9pwe9Q8fs0-yqaqpepae9pg0FirpepeKkFr0xfr-x
% fr-xb9adbaqaaeGaciGaaiaabeqaamaabaabaaGcbaGaamitaiaacI
% cacaWG4bGaaiilaiaadMhacaGGPaGaeyypa0JaamyzamaaCaaaleqa
% baGaeyOeI0IaamyAamaalaaabaGaaGOmaiabec8aWbqaaiabeU7aSj
% aadAgaaaGaaiikaiaadIhadaahaaadbeqaaiaaikdaaaWccqGHRaWk
% caWG5bWaaWbaaWqabeaacaaIYaaaaSGaaiykaaaaaaa!4A09!
L(x,y) = {e^{ - i\frac{{2\pi }}{{\lambda f}}({x^2} + {y^2})}}
\end{equation}

\section{Fresnel Propagation}
\begin{equation}
	% MathType!MTEF!2!1!+-
	% feaagCart1ev2aaatCvAUfeBSjuyZL2yd9gzLbvyNv2CaerbuLwBLn
	% hiov2DGi1BTfMBaeXatLxBI9gBaerbd9wDYLwzYbItLDharqqtubsr
	% 4rNCHbGeaGqiVu0Je9sqqrpepC0xbbL8F4rqqrFfpeea0xe9Lq-Jc9
	% vqaqpepm0xbba9pwe9Q8fs0-yqaqpepae9pg0FirpepeKkFr0xfr-x
	% fr-xb9adbaqaaeGaciGaaiaabeqaamaabaabaaGcbaGaamyvamaaBa
	% aaleaacaaIWaaabeaakiaacIcacaWG4bGaaiilaiaadMhacaGGPaGa
	% eyypa0JaeqiTdqMaaiikaiaadIhacqGHsislcaWG4bWaaSbaaSqaai
	% aad2gaaeqaaOGaaiykaiabes7aKjaacIcacaWG5bGaeyOeI0IaamyE
	% amaaBaaaleaacaWGTbaabeaakiaacMcaaaa!4AE4!
	{U_0}(x,y) = \delta (x - {x_m})\delta (y - {y_m})
\end{equation}

\begin{equation}
	% MathType!MTEF!2!1!+-
	% feaagCart1ev2aaatCvAUfeBSjuyZL2yd9gzLbvyNv2CaerbuLwBLn
	% hiov2DGi1BTfMBaeXatLxBI9gBaerbd9wDYLwzYbItLDharqqtubsr
	% 4rNCHbGeaGqiVu0Je9sqqrpepC0xbbL8F4rqqrFfpeea0xe9Lq-Jc9
	% vqaqpepm0xbba9pwe9Q8fs0-yqaqpepae9pg0FirpepeKkFr0xfr-x
	% fr-xb9adbaqaaeGaciGaaiaabeqaamaabaabaaGcbaGaamyvamaaBa
	% aaleaacaaIXaaabeaakiaacIcacaWG4bGaaiilaiaadMhacaGGPaGa
	% eyypa0ZaaSaaaeaacaWGLbWaaWbaaSqabeaacaWGPbGaaGOmaiabec
	% 8aWjaacIcacaWGMbGaey4kaSIaamOEamaaBaaameaacaWGTbaabeaa
	% liaacMcacaGGVaGaeq4UdWgaaaGcbaGaamyAaiabeU7aSjaacIcaca
	% WGMbGaey4kaSIaamOEamaaBaaaleaacaWGTbaabeaakiaacMcaaaWa
	% a8GuaeaacaWGvbWaaSbaaSqaaiaaicdaaeqaaOGaaiikaiabe67a4j
	% aacYcacqaH3oaAcaGGPaGaciyzaiaacIhacaGGWbWaaeWaaeaadaWc
	% aaqaaiaadMgacqaHapaCaeaacqaH7oaBcaGGOaGaamOzaiabgUcaRi
	% aadQhadaWgaaWcbaGaamyBaaqabaGccaGGPaaaamaabmaabaGaaiik
	% aiaadIhacqGHsislcqaH+oaEcaGGPaWaaWbaaSqabeaacaaIYaaaaO
	% Gaey4kaSIaaiikaiaadMhacqGHsislcqaH3oaAcaGGPaWaaWbaaSqa
	% beaacaaIYaaaaaGccaGLOaGaayzkaaaacaGLOaGaayzkaaGaamizai
	% abe67a4jaadsgacqaH3oaAaSqaaiabfo6atbqab0Gaey4kIiVaey4k
	% Iipaaaa!7F8A!
	{U_1}(x,y) = \frac{{{e^{i2\pi (f + {z_m})/\lambda }}}}{{i\lambda (f + {z_m})}}\iint\limits_\Sigma  {{U_0}(\xi ,\eta )\exp \left( {\frac{{i\pi }}{{\lambda (f + {z_m})}}\left( {{{(x - \xi )}^2} + {{(y - \eta )}^2}} \right)} \right)d\xi d\eta }
\end{equation}

\begin{equation}
	% MathType!MTEF!2!1!+-
	% feaagCart1ev2aaatCvAUfeBSjuyZL2yd9gzLbvyNv2CaerbuLwBLn
	% hiov2DGi1BTfMBaeXatLxBI9gBaerbd9wDYLwzYbItLDharqqtubsr
	% 4rNCHbGeaGqiVu0Je9sqqrpepC0xbbL8F4rqqrFfpeea0xe9Lq-Jc9
	% vqaqpepm0xbba9pwe9Q8fs0-yqaqpepae9pg0FirpepeKkFr0xfr-x
	% fr-xb9adbaqaaeGaciGaaiaabeqaamaabaabaaGcbaGaamyvamaaBa
	% aaleaacaaIXaaabeaakiaacIcacaWG4bGaaiilaiaadMhacaGGPaGa
	% eyypa0ZaaSaaaeaacaWGLbWaaWbaaSqabeaacaaIYaGaeqiWdaNaam
	% yAaiaacIcacaWGMbGaey4kaSIaamOEamaaBaaameaacaWGTbaabeaa
	% liaacMcacaGGVaGaeq4UdWgaaaGcbaGaamyAaiabeU7aSjaacIcaca
	% WGMbGaey4kaSIaamOEamaaBaaaleaacaWGTbaabeaakiaacMcaaaGa
	% ciyzaiaacIhacaGGWbWaaeWaaeaadaWcaaqaaiaadMgacqaHapaCae
	% aacqaH7oaBaaWaaSaaaeaacaGGOaGaamiEaiabgkHiTiaadIhadaWg
	% aaWcbaGaamyBaaqabaGccaGGPaWaaWbaaSqabeaacaaIYaaaaOGaey
	% 4kaSIaaiikaiaadMhacqGHsislcaWG5bWaaSbaaSqaaiaad2gaaeqa
	% aOGaaiykamaaCaaaleqabaGaaGOmaaaaaOqaaiaacIcacaWGMbGaey
	% 4kaSIaamOEamaaBaaaleaacaWGTbaabeaakiaacMcaaaaacaGLOaGa
	% ayzkaaaaaa!6CD0!
	{U_1}(x,y) = \frac{{{e^{2\pi i(f + {z_m})/\lambda }}}}{{i\lambda (f + {z_m})}}\exp \left( {\frac{{i\pi }}{\lambda }\frac{{{{(x - {x_m})}^2} + {{(y - {y_m})}^2}}}{{(f + {z_m})}}} \right)
\end{equation}

\begin{equation}
	% MathType!MTEF!2!1!+-
	% feaagCart1ev2aaatCvAUfeBSjuyZL2yd9gzLbvyNv2CaerbuLwBLn
	% hiov2DGi1BTfMBaeXatLxBI9gBaerbd9wDYLwzYbItLDharqqtubsr
	% 4rNCHbGeaGqiVu0Je9sqqrpepC0xbbL8F4rqqrFfpeea0xe9Lq-Jc9
	% vqaqpepm0xbba9pwe9Q8fs0-yqaqpepae9pg0FirpepeKkFr0xfr-x
	% fr-xb9adbaqaaeGaciGaaiaabeqaamaabaabaaGcbaGaamitaiaacI
	% cacaWG4bGaaiilaiaadMhacaGGPaGaeyypa0JaamyzamaaCaaaleqa
	% baGaeyOeI0IaamyAamaalaaabaGaeqiWdahabaGaeq4UdWMaamOzaa
	% aacaGGOaGaamiEamaaCaaameqabaGaaGOmaaaaliabgUcaRiaadMha
	% daahaaadbeqaaiaaikdaaaWccaGGPaaaaaaa!494D!
	L(x,y) = {e^{ - i\frac{\pi }{{\lambda f}}({x^2} + {y^2})}}
\end{equation}

\begin{equation}
% MathType!MTEF!2!1!+-
% feaagCart1ev2aaatCvAUfeBSjuyZL2yd9gzLbvyNv2CaerbuLwBLn
% hiov2DGi1BTfMBaeXatLxBI9gBaerbd9wDYLwzYbItLDharqqtubsr
% 4rNCHbGeaGqiVu0Je9sqqrpepC0xbbL8F4rqqrFfpeea0xe9Lq-Jc9
% vqaqpepm0xbba9pwe9Q8fs0-yqaqpepae9pg0FirpepeKkFr0xfr-x
% fr-xb9adbaqaaeGaciGaaiaabeqaamaabaabaaGcbaGaamyvamaaBa
% aaleaacaaIYaaabeaakiaacIcacaWG4bGaaiilaiaadMhacaGGPaGa
% eyypa0JaamyvamaaBaaaleaacaaIXaaabeaakiaacIcacaWG4bGaai
% ilaiaadMhacaGGPaGaeyyXICTaamitaiaacIcacaWG4bGaaiilaiaa
% dMhacaGGPaaaaa!49B8!
{U_2}(x,y) = {U_1}(x,y) \cdot L(x,y)
\end{equation}

\begin{equation}
% MathType!MTEF!2!1!+-
% feaagCart1ev2aaatCvAUfeBSjuyZL2yd9gzLbvyNv2CaerbuLwBLn
% hiov2DGi1BTfMBaeXatLxBI9gBaerbd9wDYLwzYbItLDharqqtubsr
% 4rNCHbGeaGqiVu0Je9sqqrpepC0xbbL8F4rqqrFfpeea0xe9Lq-Jc9
% vqaqpepm0xbba9pwe9Q8fs0-yqaqpepae9pg0FirpepeKkFr0xfr-x
% fr-xb9adbaqaaeGaciGaaiaabeqaamaabaabaaGcbaGaamyvamaaBa
% aaleaacaaIYaaabeaakiaacIcacaWG4bGaaiilaiaadMhacaGGPaGa
% eyypa0deaaaaaaaaa8qadaWcaaqaaiaadwgadaahaaWcbeqaaiaaik
% dacqaHapaCcaGGOaGaamOzaiabgUcaRiaadQhadaWgaaadbaGaamyB
% aaqabaWccaGGPaGaai4laiabeU7aSbaaaOqaaiaadMgacqaH7oaBca
% GGOaGaamOzaiabgUcaRiaadQhadaWgaaWcbaGaamyBaaqabaGccaGG
% PaaaaiGacwgacaGG4bGaaiiCamaabmaabaGaeyOeI0IaamyAamaabm
% aabaWaaSaaaeaacqaHapaCaeaacqaH7oaBcaWGMbaaaiaacIcacaWG
% 4bWdamaaCaaaleqabaWdbiaaikdaaaGccqGHRaWkcaWG5bWdamaaCa
% aaleqabaWdbiaaikdaaaGcpaGaaiyka8qacqGHRaWkdaWcaaqaaiab
% ec8aWbqaaiabeU7aSbaadaWcaaWdaeaapeGaaiikaiaadIhacqGHsi
% slcaWG4bWaaSbaaSqaaiaad2gaaeqaaOGaaiyka8aadaahaaWcbeqa
% a8qacaaIYaaaaOGaey4kaSIaaiikaiaadMhacqGHsislcaWG5bWaaS
% baaSqaaiaad2gaaeqaaOGaaiyka8aadaahaaWcbeqaa8qacaaIYaaa
% aaGcpaqaa8qacaGGOaGaamOzaiabgkHiTiaadQhadaWgaaWcbaGaam
% yBaaqabaGccaGGPaaaaaGaayjkaiaawMcaaaGaayjkaiaawMcaaaaa
% !7AC7!
{U_2}(x,y) = \frac{{{e^{2\pi (f + {z_m})/\lambda }}}}{{i\lambda (f + {z_m})}}\exp \left( { - i\left( {\frac{\pi }{{\lambda f}}({x^2} + {y^2}) + \frac{\pi }{\lambda }\frac{{{{(x - {x_m})}^2} + {{(y - {y_m})}^2}}}{{(f - {z_m})}}} \right)} \right)
\end{equation}

\begin{equation}
% MathType!MTEF!2!1!+-
% feaagCart1ev2aaatCvAUfeBSjuyZL2yd9gzLbvyNv2CaerbuLwBLn
% hiov2DGi1BTfMBaeXatLxBI9gBaerbd9wDYLwzYbItLDharqqtubsr
% 4rNCHbGeaGqiVu0Je9sqqrpepC0xbbL8F4rqqrFfpeea0xe9Lq-Jc9
% vqaqpepm0xbba9pwe9Q8fs0-yqaqpepae9pg0FirpepeKkFr0xfr-x
% fr-xb9adbaqaaeGaciGaaiaabeqaamaabaabaaGcqaaaaaaaaaWdbe
% aapaGaamyvamaaBaaaleaacaaIZaaabeaakiaacIcacaWG4bGaaiil
% aiaadMhacaGGPaGaeyypa0ZaaSaaaeaacaWGLbWaaWbaaSqabeaaca
% WGPbGaaGOmaiabec8aWjaadAgacaGGVaGaeq4UdWgaaaGcbaGaamyA
% aiabeU7aSjaadAgaaaWaa8GuaeaacaWGvbWaaSbaaSqaaiaaikdaae
% qaaOGaaiikaiabe67a4jaacYcacqaH3oaAcaGGPaGaciyzaiaacIha
% caGGWbWaaeWaaeaadaWcaaqaaiaadMgacqaHapaCaeaacqaH7oaBca
% WGMbaaamaabmaabaGaaiikaiaadIhacqGHsislcqaH+oaEcaGGPaWa
% aWbaaSqabeaacaaIYaaaaOGaey4kaSIaaiikaiaadMhacqGHsislcq
% aH3oaAcaGGPaWaaWbaaSqabeaacaaIYaaaaaGccaGLOaGaayzkaaaa
% caGLOaGaayzkaaGaamizaiabe67a4jaadsgacqaH3oaAaSqaaiabfo
% 6atbqab0Gaey4kIiVaey4kIipaaaa!7295!
{U_3}(x,y) = \frac{{{e^{i2\pi f/\lambda }}}}{{i\lambda f}}\iint\limits_\Sigma  {{U_2}(\xi ,\eta )\exp \left( {\frac{{i\pi }}{{\lambda f}}\left( {{{(x - \xi )}^2} + {{(y - \eta )}^2}} \right)} \right)d\xi d\eta }
\end{equation}

\begin{equation}
% MathType!MTEF!2!1!+-
% feaagCart1ev2aaatCvAUfeBSjuyZL2yd9gzLbvyNv2CaerbuLwBLn
% hiov2DGi1BTfMBaeXatLxBI9gBaerbd9wDYLwzYbItLDharqqtubsr
% 4rNCHbGeaGqiVu0Je9sqqrpepC0xbbL8F4rqqrFfpeea0xe9Lq-Jc9
% vqaqpepm0xbba9pwe9Q8fs0-yqaqpepae9pg0FirpepeKkFr0xfr-x
% fr-xb9adbaqaaeGaciGaaiaabeqaamaabaabaaGcbaaeaaaaaaaaa8
% qacaWGvbWaaSbaaSqaaiaaiodaaeqaaOGaaiikaiaadIhacaGGSaGa
% amyEaiaacMcacqGH9aqpdaWcaaqaaiaadwgadaahaaWcbeqaamaala
% aabaGaaGOmaiabec8aWbqaaiabeU7aSbaadaqadaqaaiaaikdacaWG
% MbGaey4kaSIaamOEamaaBaaameaacaWGTbaabeaaaSGaayjkaiaawM
% caaaaaaOqaaiaadMgacqaH7oaBcaWGMbaaaiaadwgadaahaaWcbeqa
% aiabgkHiTiaadMgadaqadaqaamaalaaabaGaaGOmaiabec8aWbqaai
% aadAgacqaH7oaBaaGaaiikaiaadIhacqGHflY1caWG4bWaaSbaaWqa
% aiaad2gaaeqaaSGaey4kaSIaamyEaiabgwSixlaadMhadaWgaaadba
% GaamyBaaqabaWccaGGPaGaey4kaSYaaSaaaeaacqaHapaCcaWG6bWa
% aSbaaWqaaiaad2gaaeqaaaWcbaGaamOzamaaCaaameqabaGaaGOmaa
% aaliabeU7aSbaacaGGOaGaamiEa8aadaahaaadbeqaa8qacaaIYaaa
% aSGaey4kaSIaamyEa8aadaahaaadbeqaa8qacaaIYaaaaSGaaiykaa
% GaayjkaiaawMcaaaaakiabg2da9maalaaabaGaamyzamaaCaaaleqa
% baWaaSaaaeaacaaIYaGaeqiWdahabaGaeq4UdWgaamaabmaabaGaaG
% OmaiaadAgacqGHRaWkcaWG6bWaaSbaaWqaaiaad2gaaeqaaaWccaGL
% OaGaayzkaaaaaaGcbaGaamyAaiabeU7aSjaadAgaaaGaamyzamaaCa
% aaleqabaGaeyOeI0IaamyAaiabfs5aenaaBaaameaacaWGTbaabeaa
% aaaaaa!8764!
{U_3}(x,y) = \frac{{{e^{\frac{{2\pi }}{\lambda }\left( {2f + {z_m}} \right)}}}}{{i\lambda f}}{e^{ - i\left( {\frac{{2\pi }}{{f\lambda }}(x \cdot {x_m} + y \cdot {y_m}) + \frac{{\pi {z_m}}}{{{f^2}\lambda }}({x^2} + {y^2})} \right)}} = \frac{{{e^{\frac{{2\pi }}{\lambda }\left( {2f + {z_m}} \right)}}}}{{i\lambda f}}{e^{ - i{\Delta _m}}}
\end{equation}

\begin{equation}
% MathType!MTEF!2!1!+-
% feaagCart1ev2aaatCvAUfeBSjuyZL2yd9gzLbvyNv2CaerbuLwBLn
% hiov2DGi1BTfMBaeXatLxBI9gBaerbd9wDYLwzYbItLDharqqtubsr
% 4rNCHbGeaGqiVu0Je9sqqrpepC0xbbL8F4rqqrFfpeea0xe9Lq-Jc9
% vqaqpepm0xbba9pwe9Q8fs0-yqaqpepae9pg0FirpepeKkFr0xfr-x
% fr-xb9adbaqaaeGaciGaaiaabeqaamaabaabaaGcqaaaaaaaaaWdbe
% aacqqHuoardaWgaaWcbaGaamyBaaqabaGccqGH9aqpdaagaaqaamaa
% laaabaGaaGOmaiabec8aWbqaaiaadAgacqaH7oaBaaGaaiikaiaadI
% hacqGHflY1caWG4bWaaSbaaSqaaiaad2gaaeqaaOGaey4kaSIaamyE
% aiabgwSixlaadMhadaWgaaWcbaGaamyBaaqabaGccaGGPaaaleaaca
% WGHbaakiaawIJ-aiabgUcaRmaayaaabaWaaSaaaeaacqaHapaCcaWG
% 6bWaaSbaaSqaaiaad2gaaeqaaaGcbaGaamOzamaaCaaaleqabaGaaG
% OmaaaakiabeU7aSbaacaGGOaGaamiEa8aadaahaaWcbeqaa8qacaaI
% YaaaaOGaey4kaSIaamyEa8aadaahaaWcbeqaa8qacaaIYaaaaOGaai
% ykaaWcbaGaamOyaaGccaGL44paaaa!6074!
{\Delta _m} = \underbrace {\frac{{2\pi }}{{f\lambda }}(x \cdot {x_m} + y \cdot {y_m})}_a + \underbrace {\frac{{\pi {z_m}}}{{{f^2}\lambda }}({x^2} + {y^2})}_b	
\end{equation}

\section{Hologram generation}
\subsection{Gratings and Lenses algorithm}

\subsection{Gerchberg-Saxton algorithm}
\subsubsection{Original algorithm in two dimensions}
The Gerchberg-Saxton algorithm is an iterative method originally developed for recovering the phase of an electron of light beam from its intensity distributions in two transverse planes. 

It can be applied to shape a light beam with calculating the phase pattern which light at one plane would require to form an almost diffraction-limited approximation to any desired intensity pattern at the second plane:


\[u_n^H=\sqrt{I_H}\exp{i\varphi_{n-1}^H}\]

\[\varphi_n^T = \arg\left(FFT\left(u_n^H\right)\right)\]

\[u_n^T=\sqrt{I_T}\exp{i\varphi_n^T}\]

\[\varphi_n^H = \arg{FFT^{-1}\left(u_n^T\right)}\]

\subsubsection{Algorithm in three dimensions}

\subsection{Comparaison}
We use the error: 
\[\epsilon_n = \sqrt{\frac{\sum_{x,y,z}\|\sqrt{I_n}-\sqrt{I_T}\|^2}{\sum_{x,y,z}I_T}}\]

\clearpage
%\bibliographystyle{plain}
%\bibliography{biblio}

\fig{slm-setup}{Optical setup used for calibrating the Spatial Light Modulator}

\fig{slm-setup3D}{3D view of the setup}

\fig{slm-interferences}{(a) Mask, (b) Interference pattern at the focal point of the lens L}

\fig{slm-shiftwide}{}

\fig{slm-shift}{}

\fig{slm-flickr}{}

\fig{slm-curve}{}

%\fig{slm-gamma}{532 nm}
\clearpage
\appendix


\backmatter
	\listoffigures
	\listoftables
\end{document}


